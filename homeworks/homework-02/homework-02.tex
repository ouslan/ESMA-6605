\documentclass[10pt, oneside]{article}
\usepackage{amsmath, amsthm, amssymb, calrsfs, wasysym, verbatim, bbm, color, graphics, geometry}

\geometry{tmargin=.75in, bmargin=.75in, lmargin=.75in, rmargin = .75in}

\newcommand{\R}{\mathbb{R}}
\newcommand{\C}{\mathbb{C}}
\newcommand{\Z}{\mathbb{Z}}
\newcommand{\N}{\mathbb{N}}
\newcommand{\Q}{\mathbb{Q}}
\newcommand{\Cdot}{\boldsymbol{\cdot}}

\newtheorem{thm}{Theorem}
\newtheorem{defn}{Definition}
\newtheorem{conv}{Convention}
\newtheorem{rem}{Remark}
\newtheorem{lem}{Lemma}
\newtheorem{cor}{Corollary}


\title{ESMA 6005: Assignacion 2}
\author{Alejandro Ouslan}
\date{Para Miercoles 4 de septiembre de 2024}

\begin{document}

\maketitle

\vspace{.25in}

\section{Problema 1}
Quiere empezar un negocio pequeño. Se ha dado cuenta de que muchos de los estudiantes de
UPRM tiene que ir a la casa de sus padres el fin de semana porque no tienen facilidades en
Mayagüez para lavar su ropa. Piensa que tener una lavandería con precios módicos y un buen
ambiente sería un buen negocio en Mayagüez. Hace una encuesta con 500 estudiantes
seleccionados aleatoriamente de la UPRM, 155 de ellos le dicen que usarían su servicio de
lavandería. Construya un intervalo de confianza de 95\% para estimar la proporción de
estudiantes de RUM que usarían su servicio lavandería.

\begin{itemize}
	\item \textbf{Ecriba el praametro de interes.}\\
	      El parametro de interes es la proporcion de estudiantes de RUM que usarian el servicio de lavanderia.
	\item \textbf{Construya el intervalo de confianza de 95\%.}
	      \[
		      \hat{p} \pm z_{\alpha/2} \sqrt{\frac{\hat{p}(1-\hat{p})}{n}} = 0.31 \pm 1.96 \sqrt{\frac{0.31(1-0.31)}{500}} = [0.27,0.35]
	      \]
	\item \textbf{Indique la interpretacion del intervalo de confianza.}\\
	      Estamos 95\% seguros de que la proporcion de estudiantes de RUM que usarian el servicio de lavanderia esta entre 27\% y 35\%.


\end{itemize}

\end{document}
