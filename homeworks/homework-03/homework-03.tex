\documentclass[10pt, oneside]{article}
\usepackage{amsmath, amsthm, amssymb, calrsfs, wasysym, verbatim, bbm, color, graphics, geometry}

\geometry{tmargin=.75in, bmargin=.75in, lmargin=.75in, rmargin = .75in}

\newcommand{\R}{\mathbb{R}}
\newcommand{\C}{\mathbb{C}}
\newcommand{\Z}{\mathbb{Z}}
\newcommand{\N}{\mathbb{N}}
\newcommand{\Q}{\mathbb{Q}}
\newcommand{\Cdot}{\boldsymbol{\cdot}}

\newtheorem{thm}{Theorem}
\newtheorem{defn}{Definition}
\newtheorem{conv}{Convention}
\newtheorem{rem}{Remark}
\newtheorem{lem}{Lemma}
\newtheorem{cor}{Corollary}


\title{ESMA 6005: Assignacion 3}
\author{Alejandro Ouslan}
\date{Para Viernes 18 de octubre de 2024}

\begin{document}

\maketitle

\vspace{.25in}

\section{Problema 1}
Se estudia el tiempo de reacción en humanos midiendo el tiempo que toma agarrar un objeto que va cayendo.
Se desea determinar si en promedio el tiempo de reacción cuando se usa la mano dominante es menor a cuando
de usa la mano no dominante. Se hizo un estudio con 12 individuos. Para cada uno de ellos se midió el tiempo
de reacción separadamente en cada mano y la mano en que primero se midió la reacción fue escogida aleatoriamente.
A continuación, el tiempo en segundos que les tomó agarrar el objeto que iba cayendo, tanto con la mano dominante
como con la mano no dominante.

\begin{itemize}
	\item \textbf{Praametro de interes.}
	      Es la diferencia en el promedio entre la mano dominante (MD) y mano no dominante (MND).
	      Esto es un problema de promedios con muestras dependientes
	\item \textbf{Construya el intervalo de confianza de 95\%.}
	      \[
		      \bar{d} \pm t_{\alpha/2} \left( \frac{s_d}{\sqrt{n}} \right) \\
		      -.012 \pm 2.201 \left( \frac{0.18}{\sqrt{12}} \right) \\
	      \]
	\item \textbf{Interpretacion del intervalo de confianza.}
	      Estamos 95\% seguros de que la diferencia entre el tiempo de reacción de la mano dominante y la mano no dominante es
	      entre -0.024 y -0.001 segundos.

	\item \textbf{hipotesis}:
	      \begin{itemize}
		      \item  $H_o: \mu_d \geq 0$
		      \item $H_a: \mu_d < 0$
	      \end{itemize}
	\item \textbf{valor critico:}
	      \[
		      \begin{split}
			      t &= \frac{\bar{d} - 0}{\frac{s_d}{\sqrt{n}}}\\
			      t &= \frac{-0.012 - 0}{\frac{0.18}{\sqrt{12}}}\\
			      t &= -2.438
		      \end{split}
	      \]
	\item \textbf{Resultasos:}
	      $statistic=-2.438, pvalue=0.0165, df=11$
	\item \textbf{Conclusión:}el valor criticó para esta prueba es 0.0165 y al ser menor que 0.05 rechazamos nuestra hipótesisnula
	      y aceptamos nuestra hipótesis alterna. En el contexto del problema tenemos suficienteevidencia para concluir que el tiempo de
	      reacción de la mano dominante es menor que la manono dominante
\end{itemize}

\end{document}
