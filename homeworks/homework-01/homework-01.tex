\documentclass[10pt, oneside]{article}
\usepackage{amsmath, amsthm, amssymb, calrsfs, wasysym, verbatim, bbm, color, graphics, geometry}

\geometry{tmargin=.75in, bmargin=.75in, lmargin=.75in, rmargin = .75in}

\newcommand{\R}{\mathbb{R}}
\newcommand{\C}{\mathbb{C}}
\newcommand{\Z}{\mathbb{Z}}
\newcommand{\N}{\mathbb{N}}
\newcommand{\Q}{\mathbb{Q}}
\newcommand{\Cdot}{\boldsymbol{\cdot}}

\newtheorem{thm}{Theorem}
\newtheorem{defn}{Definition}
\newtheorem{conv}{Convention}
\newtheorem{rem}{Remark}
\newtheorem{lem}{Lemma}
\newtheorem{cor}{Corollary}


\title{ESMA 6005: Assignacion 1}
\author{Alejandro Ouslan}
\date{Para Miercoles 4 de septiembre de 2024}

\begin{document}

\maketitle

\vspace{.25in}

\section{Problema 1}
Las botellas que produce cierta compañía embotelladora de agua anuncian en su etiqueta que
contienen 16.9 onzas fluidas. Desea tomar una muestra de al menos 100 botellas de esta
compañía para corroborar si en promedio las botellas contienen lo que se anuncia en la
etiqueta. Las botellas se almacenan en cajas de 20 botellas cada una y en el almacén de la
compañía hay 1000 cajas. En el documento adjunto (Anejo Problema 1) se muestran 4
diferentes configuraciones para el cuatro de producción de la compañía y de su almacén. Para
cada configuración, describa cómo selecciona la muestra en no más de 4 oraciones. Si su
descripción es un método de muestreo conocido, entonces indique el nombre del método de
muestreo que describió

\begin{enumerate}
	\item Para la primera configuracion, se escoge la muestra de manera aleatoria simple, es decir,
	      se escogen las 100 botellas de cualquier caja de las 1000 cajas que hay en el almacen.
	\item Para la segunda configuracion, se escoge la muestra de manera systemática, es decir, se escogen
	      seigmentos de 3 bottelas en orden esto es para tener representacion de cada linea de produccion.
	\item Para la tercera configuracion, se escoge la muestra de manera estratificada y sistematica,
	      es decir, se escogen por cada 3 cajas de cada linea de produccion una muestra aleatoria.
	\item Para la cuarta configuracion, se escoge la muestra de manera estratificada, es decir, se escogen
	      aleatoriamente las botellas por cada caja de cada linea de produccion.
\end{enumerate}

\section{Problema 2}
En clase estudiamos algunos métodos de muestreo. Lea sobre otros métodos de muestreo y
describa uno de ellos en esta tarea y una posible aplicación.

\begin{itemize}
	\item \textbf{Snowball Sampling:} Este metodo de muestreo es utilizado cuando la poblacion es dificil de
	      acceder, por lo que se empieza con un individuo y se le pide que recomiende a otros individuos que
	      cumplan con las caracteristicas de la poblacion. Este metodo es utilizado en estudios de poblaciones
	      de personas con enfermedades raras, ya que es dificil acceder a ellas.
\end{itemize}

\end{document}
